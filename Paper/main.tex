\documentclass[conference]{IEEEtran}
\IEEEoverridecommandlockouts
% The preceding line is only needed to identify funding in the first footnote. If that is unneeded, please comment it out.
\usepackage{cite}
\usepackage{amsmath,amssymb,amsfonts}
\usepackage{algorithmic}
\usepackage{graphicx}
\usepackage{textcomp}
\usepackage{xcolor}
\def\BibTeX{{\rm B\kern-.05em{\sc i\kern-.025em b}\kern-.08em
    T\kern-.1667em\lower.7ex\hbox{E}\kern-.125emX}}
\begin{document}

\title{Triggered Sentiments: NLP Insights into the Gun Control Controversy}

\author{
    \IEEEauthorblockN{
        Ojas Chaturvedi
    }
    \IEEEauthorblockA{
        \textit{
            BASIS Chandler High School
        } \\
        Chandler, USA \\
        Email: oj.chaturvedi.2024@gmail.com
    }
    % Author template
    % \and
    % \IEEEauthorblockN{
    %     2\textsuperscript{nd} Given Name Surname
    % }
    % \IEEEauthorblockA{
    %     \textit{
    %         dept. name of organization (of Aff.)
    %     } \\
    %     \textit{
    %         name of organization (of Aff.)
    %     }\\
    %     City, Country \\
    %     email address or ORCID
    % }
}

\maketitle

\begin{abstract}
    ABSTRACT WILL GO HERE
\end{abstract}

\begin{IEEEkeywords}
    INDEX TERMS WILL GO HERE
\end{IEEEkeywords}

\section{Introduction \& Literature Review}
In the United States, the debate surrounding gun control has deep historical roots, going back to the initial draft of the Bill of Rights and the addition of the Second Amendment. This amendment, which grants private citizens the right to keep and bear arms, has been debated for centuries, with pro-gun rights enthusiasts arguing that gun control legislation destroys their culture, tradition, and right to protect themselves. Pro-gun control activists argue that statistics show guns are more used for murders than self-defense (Hogan \& Rood, 2015). The urgency of this debate was highlighted in 2021 when the CDC reported that out of 26,031 murders, 81\% involved a firearm (Gramlich, 2023).

In this paper, gun control will refer to any act of regulation towards the manufacture, sale, transfer, possession, modification, or use of firearms (Kleck, 1986). This paper aims to understand the rhetorical strategies utilized in federal gun control cases. Rhetorical strategies refer to the techniques used in arguments to persuade or influence audiences, often through appeals to emotion, logic, or credibility.

Several studies have illuminated the broader discourse on gun control. For instance, a machine learning analysis of Twitter reactions to the Sandy Hook School shooting in 2012 categorized tweets into pro-gun and anti-gun sentiments, revealing the public's immediate and sustained emotional responses. Another study contrasted the polarized rhetoric of Arkansas politicians on Twitter with the more neutral framing in local newspapers, highlighting the varied platforms and tones of the debate.

However, this paper does not support the gun control debate. Instead, the purpose of this paper is to understand the rhetorical strategies employed within federal gun control cases to influence judicial opinion. While sentiment analysis has been applied to various contentious issues, such as abortion laws, the specific application to federal gun control cases remains largely uncharted. This research seeks to bridge this gap, offering a fresh perspective on the linguistic strategies that shape the gun control debate within the legal arena.

The application of sentiment analysis in the legal domain has been explored in various capacities. For instance, Liu and Chen (2017) proposed a two-phase sentiment analysis approach tailored for predicting judgments in legal cases. This approach involves different levels or methods of analysis to enhance the accuracy of predictions. These methodologies reinforce the potential of sentiment analysis in the legal sector, not just for judgment prediction but also for understanding the nuances of legal arguments and the sentiments they convey.

The gun control debate, while rooted in legal and constitutional arguments, is also deeply intertwined with societal values, cultural norms, and political ideologies. For example, different states have shown varied responses to gun-related incidents, as shown by Twitter's reactions to the Sandy Hook shooting. States with the highest gun ownership didn't necessarily show the strongest pro-gun sentiments. Instead, states like California, Texas, and New York had the most pronounced pro-gun feelings. Such findings challenge conventional wisdom and underscore the complexity of the gun control debate.

Furthermore, traditional and digital media's role in shaping public opinion cannot be understated. The study on Arkansas politicians' tweets versus local newspaper coverage revealed stark differences in framing and sentiment. While politicians employed highly polarized words resonating with their local constituents' core values, local newspapers presented a more unbiased frame. Such insights are crucial in understanding the broader ecosystem of the gun control debate and the myriad factors that influence public sentiment. The legal landscape of gun control has evolved significantly over the years. From landmark cases that have set precedents to legislative changes at both federal and state levels, the trajectory of gun control laws offers a rich tapestry of legal arguments, societal reactions, and political maneuverings. This research will delve into specific federal gun control cases, analyzing the language, tone, and sentiment of court opinions, transcripts, and legal briefs.

Moreover, the implications of this research extend beyond the academic realm. By understanding the linguistic strategies and sentiments in federal gun control cases, policymakers can make more informed decisions, legal practitioners can craft more persuasive arguments, and the general public can engage in more nuanced discussions.

In addition, the methodology employed in this research, leveraging advanced Natural Language Processing techniques, can serve as a blueprint for future studies in other domains. The potential of sentiment analysis, especially when combined with other NLP techniques, is vast and can provide invaluable insights into various topics. In conclusion, this research aims to provide a comprehensive analysis of the rhetorical strategies in federal gun control cases. By employing advanced Natural Language Processing techniques and sentiment analysis algorithms, it seeks to uncover the underlying sentiments, emotions, and linguistic patterns that have shaped the gun control debate in the legal arena over the years. The findings from this research will not only contribute to the academic discourse on gun control but also offer valuable insights for a broader audience. This endeavor hopes to foster a more informed and empathetic public discourse on a deeply divisive issue.

\section{Methodology}
I employ a mixed-methods approach through advanced Natural Language Processing (NLP) techniques. The approach will focus on sentiment, content, thematic, frequency, and cluster analysis.

The primary data used for the analysis will come from laws relating to gun control, federal gun control case transcripts, legal briefs, and court opinions from 2000 to 2022. These documents were selected using (need to come up with randomized sampling method), ensuring the dataset is representative and comprehensive of the overall field.

To prepare the data for analysis, the NLP techniques will first utilize an algorithm that conducts text normalization and anonymization, ensuring the removal of any identifiable information in accordance with ethical practices.

For the sentiment analysis, I utilize BERT (even though I finalized the analysis and ensured that it works, I haven't chosen the model yet, but they are all beneficial) for its accuracy and relevance to legal texts. BERT will assign numerical sentiment scores to the text, ranging from highly negative to highly positive, based on a 10-point scale.

For the content analysis, which will categorize the data based on the outcomes (pro-gun rights or pro-gun control), a systematic examination of case decisions and arguments will be conducted to identify recurring themes and rhetorical strategies. Key themes, rhetorical strategies, and legal reasoning will be identified and coded using a structured coding scheme. This scheme will be developed based on initial readings and refined iteratively as the analysis progresses.

To validate my findings, I will cross-reference results using multiple NLP tools and compare them against a secondary dataset.

However, I acknowledge potential limitations, such as the inherent subjectivity and biases within the NLP tools.

\end{document}
