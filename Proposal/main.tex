

% --- Set document class and font size ---

\documentclass[letterpaper, 12pt]{article}

% --- Package imports ---

% Extended set of colors
\usepackage[dvipsnames]{xcolor}

\usepackage{
  amsmath, amsthm, amssymb, mathtools, dsfont, units,          % Math typesetting
  graphicx, wrapfig, subfig, float,                            % Figures and graphics formatting
  listings, color, inconsolata, pythonhighlight,               % Code formatting
  fancyhdr, sectsty, hyperref, enumerate, enumitem, framed }   % Headers/footers, section fonts, links, lists

% lipsum is just for generating placeholder text and can be removed
\usepackage{hyperref, lipsum} 

% --- Fonts ---

\usepackage{newpxtext, newpxmath, inconsolata}

% --- Page layout settings ---

% Set page margins
\usepackage[left=1.35in, right=1.35in, top=1.0in, bottom=.9in, headsep=.2in, footskip=0.35in]{geometry}

% Anchor footnotes to the bottom of the page
\usepackage[bottom]{footmisc}

% Set line spacing
\renewcommand{\baselinestretch}{1.2}

% Set spacing between paragraphs
\setlength{\parskip}{1.3mm}

% Allow multi-line equations to break onto the next page
\allowdisplaybreaks

% --- Page formatting settings ---

% Set image captions to be italicized
\usepackage[font={it,footnotesize}]{caption}

% Set link colors for labeled items (blue), citations (red), URLs (orange)
\hypersetup{colorlinks=true, linkcolor=RoyalBlue, citecolor=RedOrange, urlcolor=ForestGreen}

% Set font size for section titles (\large) and subtitles (\normalsize) 
\usepackage{titlesec}
\titleformat{\section}{\large\bfseries}{{\fontsize{19}{19}\selectfont\textreferencemark}\;\; }{0em}{}
\titleformat{\subsection}{\normalsize\bfseries\selectfont}{\thesubsection\;\;\;}{0em}{}

% Enumerated/bulleted lists: make numbers/bullets flush left
%\setlist[enumerate]{wide=2pt, leftmargin=16pt, labelwidth=0pt}
\setlist[itemize]{wide=0pt, leftmargin=16pt, labelwidth=10pt, align=left}

% --- Table of contents settings ---

\usepackage[subfigure]{tocloft}

% Reduce spacing between sections in table of contents
\setlength{\cftbeforesecskip}{.9ex}

% Remove indentation for sections
\cftsetindents{section}{0em}{0em}

% Set font size (\large) for table of contents title
\renewcommand{\cfttoctitlefont}{\large\bfseries}

% Remove numbers/bullets from section titles in table of contents
\makeatletter
\renewcommand{\cftsecpresnum}{\begin{lrbox}{\@tempboxa}}
\renewcommand{\cftsecaftersnum}{\end{lrbox}}
\makeatother

% --- Set path for images ---

\graphicspath{{Images/}{../Images/}}

% --- Math/Statistics commands ---

% Add a reference number to a single line of a multi-line equation
% Usage: "\numberthis\label{labelNameHere}" in an align or gather environment
\newcommand\numberthis{\addtocounter{equation}{1}\tag{\theequation}}

% Shortcut for bold text in math mode, e.g. $\b{X}$
\let\b\mathbf

% Shortcut for bold Greek letters, e.g. $\bg{\beta}$
\let\bg\boldsymbol

% Shortcut for calligraphic script, e.g. %\mc{M}$
\let\mc\mathcal

% \mathscr{(letter here)} is sometimes used to denote vector spaces
\usepackage[mathscr]{euscript}

% Convergence: right arrow with optional text on top
% E.g. $\converge[p]$ for converges in probability
\newcommand{\converge}[1][]{\xrightarrow{#1}}

% Weak convergence: harpoon symbol with optional text on top
% E.g. $\wconverge[n\to\infty]$
\newcommand{\wconverge}[1][]{\stackrel{#1}{\rightharpoonup}}

% Equality: equals sign with optional text on top
% E.g. $X \equals[d] Y$ for equality in distribution
\newcommand{\equals}[1][]{\stackrel{\smash{#1}}{=}}

% Normal distribution: arguments are the mean and variance
% E.g. $\normal{\mu}{\sigma}$
\newcommand{\normal}[2]{\mathcal{N}\left(#1,#2\right)}

% Uniform distribution: arguments are the left and right endpoints
% E.g. $\unif{0}{1}$
\newcommand{\unif}[2]{\text{Uniform}(#1,#2)}

% Independent and identically distributed random variables
% E.g. $ X_1,...,X_n \iid \normal{0}{1}$
\newcommand{\iid}{\stackrel{\smash{\text{iid}}}{\sim}}

% Sequences (this shortcut is mostly to reduce finger strain for small hands)
% E.g. to write $\{A_n\}_{n\geq 1}$, do $\bk{A_n}{n\geq 1}$
\newcommand{\bk}[2]{\{#1\}_{#2}}

% Math mode symbols for common sets and spaces. Example usage: $\R$
\newcommand{\R}{\mathbb{R}}	% Real numbers
\newcommand{\C}{\mathbb{C}}	% Complex numbers
\newcommand{\Q}{\mathbb{Q}}	% Rational numbers
\newcommand{\Z}{\mathbb{Z}}	% Integers
\newcommand{\N}{\mathbb{N}}	% Natural numbers
\newcommand{\F}{\mathcal{F}}	% Calligraphic F for a sigma algebra
\newcommand{\El}{\mathcal{L}}	% Calligraphic L, e.g. for L^p spaces

% Math mode symbols for probability
\newcommand{\pr}{\mathbb{P}}	% Probability measure
\newcommand{\E}{\mathbb{E}}	% Expectation, e.g. $\E(X)$
\newcommand{\var}{\text{Var}}	% Variance, e.g. $\var(X)$
\newcommand{\cov}{\text{Cov}}	% Covariance, e.g. $\cov(X,Y)$
\newcommand{\corr}{\text{Corr}}	% Correlation, e.g. $\corr(X,Y)$
\newcommand{\B}{\mathcal{B}}	% Borel sigma-algebra

% Other miscellaneous symbols
\newcommand{\tth}{\text{th}}	% Non-italicized 'th', e.g. $n^\tth$
\newcommand{\Oh}{\mathcal{O}}	% Big-O notation, e.g. $\O(n)$
\newcommand{\1}{\mathds{1}}	% Indicator function, e.g. $\1_A$

% Additional commands for math mode
\DeclareMathOperator*{\argmax}{argmax}		% Argmax, e.g. $\argmax_{x\in[0,1]} f(x)$
\DeclareMathOperator*{\argmin}{argmin}		% Argmin, e.g. $\argmin_{x\in[0,1]} f(x)$
\DeclareMathOperator*{\spann}{Span}		% Span, e.g. $\spann\{X_1,...,X_n\}$
\DeclareMathOperator*{\bias}{Bias}		% Bias, e.g. $\bias(\hat\theta)$
\DeclareMathOperator*{\ran}{ran}			% Range of an operator, e.g. $\ran(T) 
\DeclareMathOperator*{\dv}{d\!}			% Non-italicized 'with respect to', e.g. $\int f(x) \dv x$
\DeclareMathOperator*{\diag}{diag}		% Diagonal of a matrix, e.g. $\diag(M)$
\DeclareMathOperator*{\trace}{trace}		% Trace of a matrix, e.g. $\trace(M)$
\DeclareMathOperator*{\supp}{supp}		% Support of a function, e.g., $\supp(f)$

% Numbered theorem, lemma, etc. settings - e.g., a definition, lemma, and theorem appearing in that 
% order in Lecture 2 will be numbered Definition 2.1, Lemma 2.2, Theorem 2.3. 
% Example usage: \begin{theorem}[Name of theorem] Theorem statement \end{theorem}
\theoremstyle{definition}
\newtheorem{theorem}{Theorem}[section]
\newtheorem{proposition}[theorem]{Proposition}
\newtheorem{lemma}[theorem]{Lemma}
\newtheorem{corollary}[theorem]{Corollary}
\newtheorem{definition}[theorem]{Definition}
\newtheorem{example}[theorem]{Example}
\newtheorem{remark}[theorem]{Remark}

% Un-numbered theorem, lemma, etc. settings
% Example usage: \begin{lemma*}[Name of lemma] Lemma statement \end{lemma*}
\newtheorem*{theorem*}{Theorem}
\newtheorem*{proposition*}{Proposition}
\newtheorem*{lemma*}{Lemma}
\newtheorem*{corollary*}{Corollary}
\newtheorem*{definition*}{Definition}
\newtheorem*{example*}{Example}
\newtheorem*{remark*}{Remark}
\newtheorem*{claim}{Claim}

% --- Left/right header text (to appear on every page) ---

% Do not include a line under header or above footer
\pagestyle{fancy}
\renewcommand{\footrulewidth}{0pt}
\renewcommand{\headrulewidth}{0pt}

% Right header text: Lecture number and title
\renewcommand{\sectionmark}[1]{\markright{#1} }
\fancyhead[R]{\small\textit{\nouppercase{\rightmark}}}

% Left header text: Short course title, hyperlinked to table of contents
\fancyhead[L]{\hyperref[sec:contents]{\small Bullet Points}}

% --- Document starts here ---

\begin{document}

% --- Main title and subtitle ---

\title{Bullet Points: NLP Dissection of Gun Control Discourse \\[1em]
\normalsize AP Research \& Senior Research Project}

% --- Author and date of last update ---

\author{\normalsize Ojas Chaturvedi}
\date{\normalsize\vspace{-1ex} Last updated:\today}

% --- Add title and table of contents ---

\maketitle
\tableofcontents\label{sec:contents}

% --- Main content: import lectures as subfiles ---

% TeX root = ../Main.tex

% First argument to \section is the title that will go in the table of contents. The second argument is the title that will be printed on the page.
\section[Working Title of Project]{Working Title of Project}

Provide a descriptive, yet concise title for your project. Also include your name, your BASIS advisor's name, and the name of your onsite advisor.

\subsection{Title}
% TODO: Check title
Triggered Sentiments: NLP Insights into the Gun Control Controversy

\subsection{Name}
\textbf{Name:} Ojas Chaturvedi
\newline
\textbf{Email: } oj.chaturvedi.2024@gmail.com
\newline
\textbf{Phone Number:} +1-480-572-9578

\subsection{BASIS Advisor}
\textbf{Name:} Dr. Travis May
\newline
\textbf{Email: } travis.may@basised.com

\subsection{Onsite Advisor}
% TODO: Find onsite advisor
\textbf{Name:}
\newline
\textbf{Email: }
% TeX root = ../Main.tex

% First argument to \section is the title that will go in the table of contents. The second argument is the title that will be printed on the page.
\section[Statement of Purpose]{Statement of Purpose}

Explain what you hope your research will uncover or demonstrate. State the question or series of questions you hope to answer.
% TeX root = ../Main.tex

% First argument to \section is the title that will go in the table of contents. The second argument is the title that will be printed on the page.
\section[Background]{Background}

Explain your interest in and experience with this topic. Describe any classes you have take in this or related topics, \underline{any reading you have already done in the field}, or any previous research you have conducted on this or related topics. Include any personal experience that has led you to want to do more research.
% TeX root = ../Main.tex

% First argument to \section is the title that will go in the table of contents. The second argument is the title that will be printed on the page.
\section[Significance]{Significance}

Explain why this topic is \underline{worth considering} or this question or series of questions is worth answering. What are the local or global implications of this work? \underline{Why should BASIS let} \underline{you pursue this topic?}
% TeX root = ../Main.tex

% First argument to \section is the title that will go in the table of contents. The second argument is the title that will be printed on the page.
\section[Research Methodology]{Research Methodology}

Describe the \underline{kind of research} you will conduct to complete this project (library research, internet research, interviews, observations, etc.). Explain the \underline{role your internship} will play in helping you pursue an answer to your research question. Discuss the \underline{kinds of sources} you hope to consult and \underline{the methods} you will use to extract and process the information you gather in as much detail as is possible.
% TeX root = ../Main.tex

% First argument to \section is the title that will go in the table of contents. The second argument is the title that will be printed on the page.
\section[Anticipated Problems]{Anticipated Problems}

Describe the obstacles you expect to encounter and how you hope to solve them. Try to imagine every possible problem so that you have contingency plans and the project doesn't become derailed.
% TeX root = ../Main.tex

% First argument to \section is the title that will go in the table of contents. The second argument is the title that will be printed on the page.
\section[Bibliography]{Bibliography}

List at least two sources that you consulted in order to write your proposal. Use proper bibliographic format (MLA, APA, Chicago, etc.) If you are unsure as to the appropriate ciration format, check with your BASIS advisor. \textbf{No Wikipedia!}

% --- Bibliography ---

% Start a bibliography with one item.
% Citation example: "\cite{citation}".

\begin{thebibliography}{1}

\bibitem{williams}
    Williams, David.
    \textit{Probability with Martingales}.
    Cambridge University Press, 1991.
    Print.

% Uncomment the following lines to include a webpage
\bibitem{webpage1}
  LastName, FirstName. ``Webpage Title''.
  WebsiteName, OrganizationName.
  Online; accessed Month Date, Year.\\
  \texttt{www.URLhere.com}

\end{thebibliography}

% --- Document ends here ---

\end{document}