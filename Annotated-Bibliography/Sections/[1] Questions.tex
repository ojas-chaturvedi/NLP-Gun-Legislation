% TeX root = ../Main.tex

% First argument to \section is the title that will go in the table of contents. The second argument is the title that will be printed on the page.
\section[PREP 15 Questions]{PREP 15 Questions}

% 1. Has my research question changed or been slightly reoriented in any way since September? 
% 2. Have any newer studies been published in the past three months that might be relevant to the subject of my research question?
% 3. Are there any key terms or phrases that I have become acquainted with during the process of writing the Literature Review or Methods section drafts that might yield additional sources if entered into JSTOR, EBSCO Host, or other academic journal databases? 
% 4. Do any of these newly located sources challenge any of the underlying principles of my research question or methods? How can I address that in my work?

\begin{enumerate}
  \item Has my research question changed or been slightly reoriented in any way since September?
  \item Have any newer studies been published in the past three months that might be relevant to the subject of my research question?
  \item Are there any key terms or phrases that I have become acquainted with during the process of writing the Literature Review or Methods section drafts that might yield additional sources if entered into JSTOR, EBSCO Host, or other academic journal databases?
  \item Do any of these newly located sources challenge any of the underlying principles of my research question or methods? How can I address that in my work?
\end{enumerate}

\subsection{Answers}
\begin{enumerate}
  \item Yes, my research question entirely changed since September. I started with a question about the rhetoric used in the debate surrounding the ethicality of Artificial Intelligence, but the API I was using to receieve the data was too expensive. Therefore, I switched to focusing on men's mental health, but looking at the data, I realized that it was all positive, which makes sense since no one would post about their mental health if they were feeling negative. Therefore, I switched to gun control, which is the topic I am currently working on. I am able to access data for free and there are 2 sides with very differing views.
  \item Not that I know of yet, but I will continue to look.
  \item The main key terms that I have been using in academic journal databases are ``gun control,'' ``gun rights,'' and ``gun violence''. I have also been using ``NLP'' and ``Natural Language Processing'' to find articles about the topic.
  \item None of the sources I have found so far have challenged the underlying principles of my research question or methods.
\end{enumerate}