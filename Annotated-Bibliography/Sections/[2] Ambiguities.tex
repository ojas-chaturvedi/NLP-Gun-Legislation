% TeX root = ../Main.tex

% First argument to \section is the title that will go in the table of contents. The second argument is the title that will be printed on the page.
\section[Resolving ambiguities in NLP]{A comprehensive review on resolving ambiguities in natural language processing}

% --- Bibliography ---

% Start a bibliography with one item.
% Citation example: "\cite{citation}".

% Note: The bibliography below serves as an example and is is no way shape or form related to this project.

% \bibitem{williams}
%     Williams, David.
%     \textit{Probability with Martingales}.
%     Cambridge University Press, 1991.
%     Print.

% % Uncomment the following lines to include a webpage
% \bibitem{webpage1}
%   LastName, FirstName. ``Webpage Title''.
%   WebsiteName, OrganizationName.
%   Online; accessed Month Date, Year.\\
%   \texttt{www.URLhere.com}

\begin{thebibliography}{1}

\bibitem{yadav2021comprehensive}
  A.~Yadav, A.~Patel, and M.~Shah, ``A comprehensive review on resolving ambiguities in natural language processing,'' \emph{AI Open}, vol.~2, pp. 85--92, 2021. DOI:\@ \href{https://doi.org/10.1016/j.aiopen.2021.05.001}{10.1016/j.aiopen.2021.05.001}.

\end{thebibliography}

\subsection{Summary}
This academic paper focuses on resolving ambiguities in natural language processing (NLP), a critical tool in the development of AI technologies. It focuses on how NLP tools tackle the challenge of interpreting language accurately, particularly in complex contexts like legal documents. It reviews various approaches: controlled natural languages, knowledge-based methods, checklist-based inspections, and advanced techniques (transfer learning with models like BERT). The paper highlights the difficulty of completely eliminating ambiguities due to the nuanced nature of language, pointing out both the strengths and limitations of current NLP tools. Therefore, there is a need for more refined and robust methodologies in NLP to enhance accuracy in language interpretation and disambiguation.
\subsection{Relevance to my project}
This paper will be extremely beneficial since it solves a key challenge of sentiment analysis. A simple example of this is the sentence ``The chicken is ready to eat''. This could mean two things, either the chicken is going to eat something, or the chicken is cooked and ready to be eaten. This is a very simple example, but it shows how important it is to resolve ambiguities for the NLP analysis to give results as accurately as possible. Techniques like controlled natural languages and knowledge-based methods could be instrumental in going through the legal texts. The discussion on transfer learning and BERT is particularly important, as these advanced NLP methodologies could enhance the accuracy of sentiment analysis in my research. Understanding the intricacies of language ambiguities and the tools to resolve them will be crucial in ensuring the precision and reliability of my project findings.